\documentclass[conference]{IEEEtran}
\IEEEoverridecommandlockouts
% The preceding line is only needed to identify funding in the first footnote. If that is unneeded, please comment it out.
\usepackage{cite}
\usepackage{amsmath,amssymb,amsfonts}
\usepackage{algorithmic}
\usepackage{graphicx}
\usepackage{textcomp}
\usepackage{xcolor}
\def\BibTeX{{\rm B\kern-.05em{\sc i\kern-.025em b}\kern-.08em
    T\kern-.1667em\lower.7ex\hbox{E}\kern-.125emX}}
\graphicspath{{Figures/}}

\begin{document}

\title{Secure Industrial Control System with Intrusion Detection\\
% {\footnotesize \textsuperscript{*}Note: Sub-titles are not captured in Xplore and
% should not be used}
}

\author{\IEEEauthorblockN{M Rayhan Ahmed Mithu}
\IEEEauthorblockA{\textit{Computer Science Department} \\
\textit{Tennessee Technological University}\\
Cookeville,TN,USA \\
mmithu42@students.tntech.edu}
\and
\IEEEauthorblockN{Michael Rogers, Denis Ulybyshev}
\IEEEauthorblockA{\textit{Computer Science Department, CEROC} \\
\textit{Tennessee Technological University}\\
Cookeville, TN, United States of America \\
mrogers, dulybyshev@tntech.edu}
\and
% \IEEEauthorblockN{3\textsuperscript{rd} Given Name Surname}
% \IEEEauthorblockA{\textit{dept. name of organization (of Aff.)} \\
% \textit{name of organization (of Aff.)}\\
% City, Country \\
% email address}
% \and
% \IEEEauthorblockN{4\textsuperscript{th} Given Name Surname}
% \IEEEauthorblockA{\textit{dept. name of organization (of Aff.)} \\
% \textit{name of organization (of Aff.)}\\
% City, Country \\
% email address}
% \and
% \IEEEauthorblockN{5\textsuperscript{th} Given Name Surname}
% \IEEEauthorblockA{\textit{dept. name of organization (of Aff.)} \\
% \textit{name of organization (of Aff.)}\\
% City, Country \\
% email address}
% \and
% \IEEEauthorblockN{6\textsuperscript{th} Given Name Surname}
% \IEEEauthorblockA{\textit{dept. name of organization (of Aff.)} \\
% \textit{name of organization (of Aff.)}\\
% City, Country \\
% email address}
}

\maketitle

\begin{abstract}
% Dr. Rogers - pls add a paragraph 
To detect a root cause of anomaly, it is essential to provide confidentiality and integrity for log records. In this paper, we present a secure data container that can be used to protect log records as well as detected state for devices in cyber-physical systems. Data protection is provided in transit and at rest. Our data container supports role-based and attribute-based access control, so that every party can access only those data subsets for which the party is authorized. Additionally, data container provides data leakage prevention. Furthermore, we address the insider threat by detecting several types of data leakages that can be made by malicious insiders. 
\end{abstract}

\begin{IEEEkeywords}
intrusion detection, data privacy, industrial control systems 
\end{IEEEkeywords}

\section{Introduction}
Industrial control system (ICS) is the technology used to automate the manufacturing process. It is responsible for controlling and managing large number of field devices. The manufacturing industry has seen an astronomical rise in the adoption of ICS in recent years. The fierce competition among the companies has been the main catalyst for this revolution in the manufacturing industry. According to the study report published by  Market Research Future (MRFR), the industrial control system market will be growing even more in the future \cite{c1}. A lot of these industrial control system are deployed in critical infrastructures such as the smart grid, health care system, water purification, nuclear plants. As a result, any kind of security breach and attack can have significant impact on human lives as well as be very costly. Securing these system has become a priority as a result of the increasing number of attacks in these domain. 
\section{Related Work}

Concerns over the security of industrial control system has led many researchers to look for different security solutions to protect the network. For this paper we focus on the software solutions that are available in the state of the art. Intrusion detection is one of the most popular techniques used to prevent and detect unauthorized action. The same technique with various approaches are proposed by many researchers to use in the industrial network. 
\par In \cite{c2}, the authors proposed an intrusion detection system that focuses on devices that are connected with serial link. The authors proposed a Snort based intrusion detection system for MODBUS RTU/ASCII communication. The MODBUS traffic is converted to MODBUS TCP/IP and sent to Snort for rule matching. Snort was implemented in two setup, one was in passive mode where the traffic is only monitored by Snort and in inline setup includes rules for dropping packets.
\par A multi agent intrusion detection system was proposed by Tsan and Kwong \cite{c3} in industrial network. They implemented the unsupervised Ant Colony Clustering Model (ACCM) in the "decision agent" that is responsible for learning and classification of prepossessed data. This implementation shows the improved performance ant-based clustering. The KDD-Cup99 dataset was used for the test evaluation with different clustering algorithms and feature extraction techniques. K-Means clustering achieves 89.17\% average detection rate and  4.29\% false positive rate with the Fast Independent Component Analysis (FastICA) feature extraction.
\section{Security in ICS}
Industrial control system was initially designed to operate within a close environment and manage the overall system. However, this has changed as the ICS network is now globally connected via the internet technology. The global connection has a lot of benefits that includes managing multiple field sites with thousands of devices in different location from one control center, faster data access and and inseparability among devices with different communication protocols. It also exposed the industrial control system to all the vulnerabilities and security threats that exists in the commercial network.Communication among the main components such as programmable logic controller (PLC), supervisory control and data acquisition (SCADA), Control devices is what makes this whole system functional. The communication protocols for these components are not similar to the typical commercial network protocols. Security was not a big concern when these protocols were designed as the system was kept isolated from the outside world. 
\begin{figure}[htbp]
\centering
\centerline{\includegraphics [width=9cm] {Architecture1.png}}
\caption{Industrial Control System Architecture}
\label{fig}
\end{figure}


\subsection{Software Solutions}\label{AA}
\subsection{Observed Gap}
Even though there are a lot of approaches for intrusion detection in ICS are present in the literature, we observe that most of these solutions focus on the network traffic data generated in the control system.  
\section{Proposed Architecture}
Intrusion detection system is a great method for attack detection and prevention in the commercial network by monitoring network traffic with rule based or anomaly based approach. However, in the industrial network it is not always enough to just monitor the network traffic. It is possible that a sequence of valid commands can lead to a critical situation. As a result, a lot of false positives and false negatives are generated by the IDS. We propose a novel approach where the IDS can validate network traffic with the help from the device generating the traffic. A device state information equivalent to each network packet is created and stored. The device state information is kept in a secure container which can be access upon request. The architecture has three major system components that is focused on completing the steps. Each step will which are:
\begin{enumerate}
\item Off-line Tools
\item SCADA components
\item Secure Container
\item Detection Process
\subsection{Off-line Tools}
This system component of the architecture is responsible to identify important information about the state of the device and the location of those state information on the device. We call this component the off-line tools as these will be executed when the SCADA system is not running real time. We propose three modules for this component. 
\par The first module is called the \textit{\textbf{Image Mapping Module}}, this module is responsible for identifying the important information to represent the device state and gather those information.
\par The second module will be the \textit{\textbf{Golden Image Module}} which will use the information collected by the image mapping module and create the golden image of the device system which is considered to be safe and standard. \par The third module in this component is the \textit{\textbf{Training Module}}, this module will be used to help the detection process of any malicious activity. The training module will be fed data collected from the SCADA environment with different device state and and mapped with the golden image to learn more about malicious states of the device. This information will be provided to the the detection process to identify a critical or bad state of the device more accurately and fast.
\end{enumerate}
\subsection{SCADA Components}
The second component of our architecture is the SCADA components. This component of the system will monitor system states real time and transfer the state information to the detection process. Two modules are required to execute these actions. The first module is the \textit{\textbf{State Monitor}} and the other module is called \textit{\textbf{State Router}}.
\par The state monitor module is responsible for real time monitoring of the device state. The information collected by the image mapping module in the off-line tools component is used in this module to gather the important sate information from the device. State monitor module will have prior knowledge of what information are necessary to represent the state and where the information are kept in the device. 
\par The device state information collected by the state monitor module will be transferred by the state router module to the node that is computationally more resourceful. This node can perform security analysis to identify if the device is compromised.
\subsection{Secure Data Container}
Proposed solution for protecting data in transit and at rest with providing leakage detection, as well as role-based and attribute-based access control, relies on Secure Mobile Protected Agent with Data (SMPAD). The idea was inspired by an Active Bundle concept [2], [3]. SMPAD is a self-protecting data container that incorporates sensitive data in encrypted form with watermarks, access control policies, metadata, provenance data collector, policy and attribute enforcement kernel. SMPAD can be used to store log files as well as sensor data snapshots. Each separate data subset is encrypted with separate encryption key, generated on-the-fly for each client’s role based on the unique information from the SMPAD execution control flow path. The access control model relies on role-based and attribute-based access control model and guarantees that a client will be able to access only those data subsets from a data container for which the client is authorized [4]. Data container provides tamper-resistance and data integrity. Moreover, SMPAD can detect several types of data leakages that can be made by authorized insiders [1]. The novelty of the proposed solution is that it works in both centralized and decentralized peer-to-peer network architectures. Central Authority (CA) is not required neither for data recipient’s key generation nor for access control policy enforcement.

\begin{figure}[htbp]
\centering
\centerline{\includegraphics [width=6cm] {SMPAD.png}}
\caption{SMPAD: Secure Data Container}
\label{fig}
\end{figure}

%Denis:
- list set of supported attributes for ABAC
- talk about on-the-fly key generation scheme
- talk about data leakage detection via watermarks
 
\subsection{Detection Process}
Detection process is the most important component of the secure architecture. This process is responsible for intrusion detection in the SCADA environment. Typical intrusion detection system (IDS) uses the network packets to analyze the traffic and generate alert in case of any abnormal activity in the network. However, in SCADA environment, it is not necessary that an abnormal network behavior will always be the reason for intrusion. Often times a series of valid commands can lead to a critical situation in the SCADA environment that will cause serious damage to the infrastructure. The IDS will not generate any alert in that situation as the network traffic will not look suspicious or abnormal.
\par The intrusion detection system in this component uses the information gathered by the training module in the off-line tools component to learn about device state. The IDS also receives real time information from the state router and performs analysis to detect compromised device. The information from the state router is also used to improve the IDS detection engine based on the accuracy of intrusion detection. The main idea in the detection process is to provide additional information to the IDS along with the network packets to improve the detection method. The device state information will help the IDS to verify any alert generated from analyzing the network traffic of the SCADA system.  
\begin{figure}[htbp]
\centering
\centerline{\includegraphics [width=.5\textwidth]{sec_arch.png}}
\caption{Secure Industrial Control System Architecture}
\label{fig}
%
\end{figure}
\section{Work in Progress and Future Direction}

\section{Conclusion}




\begin{thebibliography}{00}
\bibitem{c1}Industrial Control Systems (ICS) Market 2018 Global Analysis, Industry Size, Share Leaders, Current Status by Major Key vendors and Trends by Forecast to 2023. https://www.marketwatch.com/press-release/industrial-control-systems-ics-market-2018-global-analysis-industry-size-share-leaders-current-status-by-major-key-vendors-and-trends-by-forecast-to-2023-2018-11-29
\bibitem{c2}Morris, T., Vaughn, R., \& Dandass, Y. (2012). A retrofit network intrusion detection system for MODBUS RTU and ASCII industrial control systems. Proceedings of the Annual Hawaii International Conference on System Sciences, 2338–2345. https://doi.org/10.1109/HICSS.2012.78
\bibitem{c3}Tsang, C. H., \& Kwong, S. (2005). Multi-agent intrusion detection system in industrial network using ant colony clustering approach and unsupervised feature extraction. Proceedings of the IEEE International Conference on Industrial Technology, 2005, 51–56. https://doi.org/10.1109/ICIT.2005.1600609
\bibitem{c4}Kiss, I., Genge, B., Haller, P., & Sebestyen, G. (2014). Data clustering-based anomaly detection in industrial control systems. Proceedings - 2014 IEEE 10th International Conference on Intelligent Computer Communication and Processing, ICCP 2014, 275–281. https://doi.org/10.1109/ICCP.2014.6937009
\end{thebibliography}
\vspace{12pt}
\end{document}
